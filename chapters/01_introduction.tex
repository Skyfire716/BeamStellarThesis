% !TeX root = ../main.tex
% Add the above to each chapter to make compiling the PDF easier in some editors.

\chapter{Introduction}\label{chapter:introduction}
Virtual Reality Headsets and other Head Mounted Displays arrived at the consumer market in the year 2016.
Since then many global mega corporations took their shoot to conquer this market and developed many HMDs.
We have seen cable bounded models where computation can be done on a high power stationary PC and battery life is no limitation for the experience but cables limit the movement range of the user and might oppose a danger of tripping over it.
Some approaches used gaming laptops in a backpack to avoid the cable issues but had some loss in image quality due to less computational resoureces available on the mobile platform and limited battery life to support everything.
Also these backpacks were pretty heavy and exausting for users.
A newer approach is to include highly efficient smartphone chips of custom made silicon into the headsets to improve battery life and reduce the physical size of the computational side.
This approach also limits the apps and games that the headset is capable of running at a decent framerate.
Another approach was a wireless connection to stream the video to the headset from a stationary PC but this requires a sophisticated highsped wireless bandwidth which might compete with WIFI and Bluetooth devices for time on the medium and thus might suffer from reduced throughput or instabilities.
Altough all their efforts non of the models could achieve high adoption.
REFERENCE STEAM HARDWARE SURVEY AND STATISTA SELL NUMBERS
Manufacturers put a lot of effort to improve the Headset ergonomics and improve the user experience.
Displays improved, headset Weight was reduced and better distributed, motion sickness was studied and prevention tactics strengthend.
Anyways broad adoption by mainstream users still lacks behind.
This raises the question if the current headset approaches for the VR headset technology can overcome the challenges that keep users from using this technology or if we need a new approach to give users a VR or AR experience.
A key aspect for VR and AR experiences is that users percieves a virtual 3D enviornment or the real world is overlayd with additional graphical information that adapts and follows the users head and body movement.
Current approaches do this by mounting displays right infront of the users eyes and constraint them to their head.
Some devices in the AR field use semi transparent mirrors to redirect a displays light in the optical path of the eyes.
All these approaches add a lot of technology to the user himself and try to optimize for some aspects in the weight, graphical result and battery life triangle.
Beaming Displys are a new concept for head mounted displays that try to remove as much technology as possible from the user.
The Beaming Display concept is relativly new and foundations have to be layed there is no commercial available product yet.
This concept uses a projector and a steering mirror that are mounted somewhere in a room.
A user wears a minimalistic pair of glasses that allow for tracking the user and receiving the light from the projector without harming the users eyes.
Any computation can be done stationary with access to a wall outlet and the hardware can be as capable and size demanding as desired.



This thesis evaluates a fast 6-DoF Tracking and Pose estimation system for a Beaming Display, a special kind of Head Mounted Displays (HMD).
Specifically XXX well established Perspective-n-Point (PnP) pose computation algorithms are implemented in the programmable logic (PL) of a Field Programmable Gate Array (FPGA) for highspeed HMD tracking and projection correction.
The resulting pose estimations are validated against ground truth data provided by an outside in tracking sysetem.
Also a fesible application volume for the Beaming Dispay station is narrowed down.

\section{Section}
Citation test~\parencite{latex}.

\subsection{Subsection}


